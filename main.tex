%  template
\def\maindoc{}%{{{

\documentclass[a4paper,12pt]{article}
\usepackage[UTF8]{ctex} % 支持中文
\usepackage{amsmath,graphicx,multirow}

\usepackage[left=0.5in,right=0.5in,top=1in,bottom=1in]{geometry}

\usepackage{fancybox,xcolor,times,listings,fancyhdr,titlesec}

\pagestyle{fancy}
\renewcommand{\sectionmark}[1]{\markright{#1}}
\renewcommand{\subsectionmark}[1]{\markright{#1}}
\fancyhf{}
\fancyfoot[C]{第 \thepage\ 页,共 \pageref{LastPage} 页}
\fancyhead[R]{\rightmark}
\fancyhead[L]{\bfseries Sshwy}
\renewcommand{\headrulewidth}{0.4pt} % 注意不用 \setlength
\renewcommand{\footrulewidth}{0pt}

\setmonofont{mononoki}
%代码设置
\lstset{ %{{{
    language=c++,                   % the language of the code
    %basicstyle=\ttfamily,           % the size of the fonts that are used for the code
    basicstyle=\fontsize{10}{10}\selectfont\ttfamily,
    numbers=left,                   % where to put the line-numbers
    numberstyle=\color{gray},  % the style that is used for the line-numbers
    %stepnumber=2,                   % the step between two line-numbers. If it's 1, each line 
    % will be numbered
    %numbersep=5pt,                  % how far the line-numbers are from the code
    backgroundcolor=\color{white},      % choose the background color. You must add \usepackage{color}
    showspaces=false,               % show spaces adding particular underscores
    showstringspaces=false,         % underline spaces within strings
    showtabs=false,                 % show tabs within strings adding particular underscores
    frame=l,                   % adds a frame around the code
    rulecolor=\color{black},        % if not set, the frame-color may be changed on line-breaks within not-black text (e.g. commens (green here))
    tabsize=2,                      % sets default tabsize to 2 spaces
    captionpos=b,                   % sets the caption-position to bottom
    breaklines=true,                % sets automatic line breaking
    breakatwhitespace=false,        % sets if automatic breaks should only happen at whitespace
    title=\lstname,                 % show the filename of files included with \lstinputlisting;
    % also try caption instead of title
    keywordstyle=\color{blue},          % keyword style
    commentstyle=\color{gray},       % comment style
    stringstyle=\color{magenta},         % string literal style
    escapeinside={\%*}{*)},            % if you want to add LaTeX within your code
    morekeywords={*,...}               % if you want to add more keywords to the set
}%}}}
%}}}
\begin{document}

\title{Algorithms for XCPC}
\author{Sshwy}
\date{\today}
\maketitle

\newpage

\pagenumbering{roman}
\tableofcontents
\newpage
\pagenumbering{arabic}

\section{代码头}

\lstinputlisting[language=c++]{code/head.cpp}

\section{字符串}

\subsection{KMP算法}
\lstinputlisting[language=c++]{code/kmp.cpp}
\subsection{Manacher算法}
\lstinputlisting[language=c++]{code/manacher.cpp}
\subsection{后缀数组}
\lstinputlisting[language=c++]{code/sa.cpp}
\subsection{后缀自动机}
\lstinputlisting[language=c++]{code/sam.cpp}
\subsection{广义后缀自动机}
\lstinputlisting[language=c++]{code/general_sam.cpp}
\subsection{回文自动机}
\lstinputlisting[language=c++]{code/pam.cpp}

\section{数论与线性代数}

\subsection{EX-BSGS算法}
\lstinputlisting[language=c++]{code/exbsgs.cpp}
\subsection{Pollard-Rho和Miller}
\lstinputlisting[language=c++]{code/math.cpp}
\subsection{线性基}
\lstinputlisting[language=c++]{code/linear_basis.cpp}
\lstinputlisting[language=c++]{code/linear_basis2.cpp}
\lstinputlisting[language=c++]{code/linear_basis3.cpp}
\subsection{Min 25}
\lstinputlisting[language=c++]{code/min_25.cpp}
\subsection{Min 25 杰哥}
\lstinputlisting[language=c++]{code/min_25_jie.cpp}
\subsection{二次剩余 Cipolla}
\lstinputlisting[language=c++]{code/cipolla.cpp}
\subsection{特征多项式}
\lstinputlisting[language=c++]{code/characteristic_polynomial.cpp}
\subsection{中国剩余定理 \& exgcd}
\lstinputlisting[language=c++]{code/crt.cpp}
\subsection{类欧几里得算法}
\lstinputlisting[language=c++]{code/euclideanoid.cpp}
\subsection{自然数幂和}
\lstinputlisting[language=c++]{code/faulhaber.cpp}

\section{多项式相关}

\subsection{FFT}
\lstinputlisting[language=c++]{code/fft.cpp}
\subsection{NTT}
\lstinputlisting[language=c++]{code/ntt.cpp}
\subsection{FWT}
\lstinputlisting[language=c++]{code/fwt.cpp}
\subsection{全家桶}
\lstinputlisting[language=c++]{code/poly.cpp}

\section{图论}

\subsection{MCMF 最大费用最大流}
\lstinputlisting[language=c++]{code/mcmf.cpp}
\subsection{DINIC 算法求最大流}
\lstinputlisting[language=c++]{code/dinic.cpp}
\subsection{朱刘算法}
\lstinputlisting[language=c++]{code/dmst.cpp}
\subsection{KM 算法}
\lstinputlisting[language=c++]{code/km.cpp}
\subsection{Tarjan SCC}
\lstinputlisting[language=c++]{code/tarjan_scc.cpp}

\section{数据结构}

\subsection{左偏树}
\lstinputlisting[language=c++]{code/leftist.cpp}
\subsection{Splay}
\lstinputlisting[language=c++]{code/splay.cpp}
\subsection{非旋转Treap}
\lstinputlisting[language=c++]{code/FHQ_treap.cpp}
\subsection{LCT}
\lstinputlisting[language=c++]{code/lct.cpp}
\subsection{点分治}
\lstinputlisting[language=c++]{code/centroid_decomposition.cpp}
\subsection{笛卡尔树}
\lstinputlisting[language=c++]{code/cartesian_tree.cpp}
\subsection{树链剖分}
\lstinputlisting[language=c++]{code/heavy_light_decomposition.cpp}
\subsection{长链剖分 \& K 级祖先}
\lstinputlisting[language=c++]{code/kth_ancestor.cpp}
\subsection{虚树}
\lstinputlisting[language=c++]{code/virtual_tree.cpp}

\section{其他}

\subsection{计算几何}
\lstinputlisting[language=c++]{code/geometry.cpp}
\subsection{欧拉序求LCA}
\lstinputlisting[language=c++]{code/euler_lca.cpp}
\subsection{IO优化}
\lstinputlisting[language=c++]{code/io.cpp}


\label{LastPage}
\end{document}
